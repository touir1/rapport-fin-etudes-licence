\addcontentsline{toc}{chapter}{Introduction générale}
\chapter*{Introduction générale}
Le domaine de gestion de projet grandi de plus en plus et ceci à cause d'un besoin critique qui est présent depuis l'apparition des civilisations. Quel que soit le projet, vous rencontrerez toujours les mêmes contraintes :\\
\begin{itemize}
    \item[$\bullet$] Qualité: votre produit doit répondre aux attentes
    \item[$\bullet$] Délai: votre produit doit être livré dans les délais
    \item[$\bullet$] Coûts: la réalisation du projet ne doit pas dépasser les limites budgétaires.\\
\end{itemize}

Ces derniers sont liés et chaque action sur l'un entraîne des répercussions sur les autres. Si on réduit les délais, il faut soit augmenter les coûts en mettant plus de mains-d'œuvre sur le projet, soit de réduire la qualité du produit pour finir dans les temps. Et ceci sans parler des imprévus. Sur ce, il faut une excellente planification du projet et une bonne gestion de risque pour le bon déroulement du projet.\\

Un chef de projet est un membre de l'équipe qui gère tous ces contraintes. Pour cela, il doit non seulement avoir des connaissances techniques de base, mais en plus il doit connaître les compétences des membres de son équipe. Et ceci pour pouvoir calculer le temps nécessaire pour finir un scénario ou une tâche dans un projet et pouvoir distribuer les tâches sur les membres.\\

Dans le cadre de mon projet de fin d'études et dans cet aspect, Linedata m'a proposé le sujet intitulé "Tableau de bord de suivi et automatisation des plannings d'un projet". Cette dernière a voulu faciliter la tâche d'un chef d'équipe en automatisant quelques traitements qu'il fait et en mettant à sa disposition des tableaux de bords pour avoir une vue globale sur le rendement de son équipe.\\

Notre projet de fin d'études comporte les chapitres suivants:\\
\begin{itemize}
    \item[$\circ$] Le premier chapitre "Cadre général du projet", présente l'organisme d'accueil, la méthodologie de travail utilisé et le contexte du projet.
    \item[$\circ$] Le deuxième chapitre "Analyse des besoins", présente les acteurs et besoins du projet en plus du backlog produit et la planification de release.
    \item[$\circ$] Le troisième chapitre "Conception et réalisation de l'application", présente l'environnement de travail et les choix architecturaux.
    \item[$\circ$] Le quatrième chapitre "Organisation des sprints", présente la réalisation du projet.\\
\end{itemize}
Et nous finirons le rapport par une conclusion générale et la bibliographie.